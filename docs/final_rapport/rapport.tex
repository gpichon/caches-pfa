\documentclass[a4paper]{report}
\usepackage[utf8]{inputenc}
\usepackage[T1]{fontenc}
\usepackage[french]{babel}
\usepackage{geometry}
\geometry{hmargin=3.5cm,vmargin=3.5cm}
\usepackage{graphicx}
\usepackage{fancyhdr}
\pagestyle{fancy}

\lhead{Rapport}
\rhead{Simulateur de caches multi-c\oe ur}
\lfoot{ENSEIRB-MATMECA}
\rfoot{PFA 2013-2014}

\begin{document}

\thispagestyle{empty}

\vspace{\stretch{1}}
\hrule
\begin{flushleft}
\huge{Simulateur de caches sur\\architecture multi-c\oe ur :}\\
\end{flushleft}
\begin{flushright}
\Huge\textbf{Cahier des charges}\\
\end{flushright}
\hrule

\vspace{\stretch{1}}
\noindent\textbf{Auteurs :}
\emph{DUBOIS Nicolas, GOUDET Pierre, HENG Nicolas, HONORAT Alexandre, MARAIT Gilles, PICHON Grégoire}\\
\\
\noindent\textbf{Client :}
\emph{M. BARTHOU Denis}\\
\\
\noindent\textbf{Responsable pédagogique :}
\emph{M. MORANDAT Floréal} 

\vspace{\stretch{1}}
\normalsize
\begin{center}
  Deuxième année, filière informatique\\
  Date : \today\\
  \textsc{Enseirb-Matmeca}
\end{center}


\newpage
\tableofcontents

\renewcommand{\labelitemi}{$\bullet$}

\newpage
\chapter{Introduction}

\section{Généralités sur les caches}

Jusqu'aux années 2000, la vitesse des processeurs a considérablement augmenté (loi de Moore) alors que le temps d'accès à la mémoire RAM (Random-Access Memory) est resté globalement le même. Pour permettre au processeur d'accèder plus rapidement à des éléments mémoire, des caches sont donc utilisés entre les processeurs et la RAM. Cela permet de hiérarchiser la mémoire, avec des éléments dont le temps d'accès est différent. Cette organisation hiérarchique de la mémoire présente plusieurs objectifs : \\
\begin{itemize}
\item permettre de contenir un nombre conséquent de données ;
\item être organisée de manière à être rapide ;
\item ne pas coûter trop cher.\\ 
\end{itemize}

Les caches permettent de stocker la mémoire utilisée récemment par le processeur, en se basant sur deux concepts: la localité spatiale et la localité temporelle. La localité temporelle stipule qu'une cellule mémoire accédée récemment sera très probablement utilisée dans un futur proche. La localité spatiale est l'idée que si l'on accède à une cellule mémoire $X$, la cellule mémoire $X+1$ a de grandes chances d'être utilisée. \\

Les mémoires de haut niveau -- les plus proches du processeur sont notées L1 -- sont généralement de petite taille. Leur coût est conséquent mais leur accès est très rapide. Il existe plusieurs niveaux de caches, $3$ dans la plupart des architectures récentes. Les L1 sont très rapides mais peuvent contenir peu de données (de l'ordre de la dizaine de Ko) alors que le L3 est plus lent mais contient beaucoup de données (de l'ordre du Mo). On peut résumer comme par la figure \ref{img:hierarchy} une hiérarchie mémoire classique.\\

\begin{figure}[!h]
\begin{center}
   \includegraphics[scale=0.75]{hierarchy.png}
   \caption{\label{img:hierarchy} Hiérachie mémoire}
\end{center}
\end{figure}

==> Préciser source de l'image + légende\\

De cette manière, quand le processeur utilise une donnée qui est dans le cache, il fait un \textit{hit} : il n'a pas besoin de la charger à partir de la mémoire principale. Lorsque la donnée dont il a besoin n'est pas présente dans le cache, il fait un \textit{miss} et le coût d'accès à la donnée est beaucoup plus important. \\

Les données mémoire vont être lues via des \textit{load} et écrites via des \textit{store}. Lorsqu'une donnée est modifiée, il est nécessaire de propager les modifications jusqu'à la mémoire principale (c'est donc un \textit{store}), afin de garantir la cohérence du système. Une politique basique, le \textit{write-through}, serait de réécrire directement en mémoire après toute modification. Dans la suite de ce document, nous considérerons uniquement la politique \textit{write-back}, qui permet de retarder la réecriture tant que la donnée n'est pas lue pas une autre entité que celle qui l'a modifiée.

\section{Création d'un simulateur de caches}

Décrire ici notre projet. Partie copiée du cahier des charges éventuellement.


\newpage
\chapter{Généralités sur les caches}
Le but de cette partie est de résumer un certain nombre de techniques relatives à la gestion de la hiérarchie des caches. Le comportement général d'un cache sera explicité, avant d'étudier les moyens mis en {\oe}uvre afin d'assurer la cohérence de l'ensemble des caches. Pour finir certaines subtilités qui différencient le comportement global des caches seront étudiées.

\subsection{Utilité des caches}

Jusqu'aux années 2000, la vitesse des processeurs a considérablement augmenté (loi de Moore) alors que le temps d'accès à la mémoire RAM (Random-Access Memory) est resté globalement le même. Pour permettre au processeur d'accèder plus rapidement à des éléments mémoire, des caches sont donc utilisés entre les processeurs et la RAM. Cela permet de hiérarchiser la mémoire, avec des éléments dont le temps d'accès est différent. Cette organisation hiérarchique de la mémoire présente plusieurs objectifs : \\
\begin{itemize}
\item permettre de contenir un nombre conséquent de données ;
\item être organisée de manière à être rapide ;
\item ne pas coûter trop cher.\\ 
\end{itemize}

Les caches permettent de stocker la mémoire utilisée récemment par le processeur, en se basant sur deux concepts: la localité spatiale et la localité temporelle. La localité temporelle stipule qu'une cellule mémoire accédée récemment sera très probablement utilisée dans un futur proche. La localité spatiale est l'idée que si l'on accède à une cellule mémoire d'adresse $X$, la cellule mémoire d'adresse $X+1$ a de grandes chances d'être utilisée. \\

Les mémoires de bas niveau -- les plus proches du processeur -- sont notées L1 et sont généralement de petite taille. Leur coût est conséquent mais leur accès est très rapide. Il existe plusieurs niveaux de caches, $3$ dans la plupart des architectures récentes. Les L1 sont très rapides mais peuvent contenir peu de données (de l'ordre de la dizaine de Ko) alors que le L3 est plus lent mais contient beaucoup de données (de l'ordre du Mo). La figure \ref{img:hierarchy} résume une hiérarchie mémoire classique.\\

\begin{figure}[!h]
\begin{center}
   \includegraphics[scale=0.75]{images/hierarchy.png}
   \caption{\label{img:hierarchy} Hiérachie mémoire - www.pps.univ-paris-diderot.fr}
\end{center}
\end{figure}

De cette manière, quand le processeur utilise une donnée qui est dans le cache, il fait un \textit{hit} : il n'a pas besoin de la charger à partir de la mémoire principale. Lorsque la donnée dont il a besoin n'est pas présente dans le cache, il fait un \textit{miss} et le coût d'accès à la donnée est beaucoup plus important. Si pour un programme donné, il y a 10\% de \emph{misses}, le temps total d'execution sera le double qu'avec 100\% de \emph{hits}. \\

Les données mémoire vont être lues via des \textit{load} et écrites via des \textit{store}. Lorsqu'une donnée est modifiée, il est nécessaire d'informer des modifications jusqu'à la mémoire principale (c'est donc un \textit{store}), afin de garantir la cohérence du système. Une politique basique, le \textit{write-through}, serait de réécrire directement en mémoire après toute modification. Dans la suite de ce document, nous considérerons uniquement la politique \textit{write-back}, qui permet de retarder la réecriture tant que la donnée n'est pas lue par une autre entité que celle qui l'a modifiée.

\section{Principes généraux d'un cache}
Cette section entend préciser le fonctionnement général d'un cache : comment il est possible d'y ajouter une donnée, quelle est la correspondance entre les blocs de la mémoire principale et les lignes de cache ou encore comment une donnée peut être evincée d'un cache. \\

Un cache possède un ensemble de \textit{sets} contenant chacun un ensemble de lignes. Le nombre de lignes par \textit{set} est appelé l'associativité du cache. Au chargement d'une donnée, elle est insérée dans une ligne vide du \textit{set} correspondant à sa plage d'adresse. Les données qui suivent (dans la mémoire supérieure) la donnée voulue sont aussi recopiées afin de remplir totalement la ligne -- généralement de $32$ ou $64$ octets -- et exploiter le principe de localité spatiale.

\subsection{\'Etiquettes}
Pour pouvoir retrouver une donnée dans le cache, une table d'étiquettes est tenue à jour. Concrètement, une adresse mémoire est séparée en trois champs comme sur la figure suivante : \\

\begin{figure}[!h]
\begin{center}
   \includegraphics[scale=0.50]{images/etiquette.jpeg}
   \caption{\label{img:etiquette} Adresse mémoire - common.wikimedia.org}
\end{center}
\end{figure}

Les étiquettes sont stockées dans une table des étiquettes, elles serviront a identifier les différents blocs mémoires pouvant être au même endroit dans un cache. L'index correspond au numéro de \textit{set} dans lequel se trouve la ligne de cache. Pour finir l'offset correspond au bloc dans la ligne de cache.

\subsection{Fonction de correspondance}
Un cache de taille $n$ contient un ensemble $p$ de lignes de taille $m$, tel que $n = p \times m$. Afin de placer et récupérer une donnée dans le cache, une fonction de correspondance avec la mémoire est nécessaire. Il existe trois cas de figure. Pour la suite, nous prendrons : \\
\begin{itemize}
\item $i$ le numéro de \textit{set} du cache
\item $j$ le numéro du bloc mémoire
\item $s$ le nombre de \textit{sets} du cache
\item $k$ l'associativité du cache 
\end{itemize}

\subsubsection{\emph{Direct associative}}
Un cache est en correspondance directe si à chaque bloc mémoire est associé une unique ligne du cache. Le nombre de \textit{sets} $s$ du cache est alors égal à son nombre de lignes, $p$. Lorsqu'un bloc mémoire $j$ est ajouté dans le cache, la ligne correpondante est chargée à la ligne $i = j\ modulo\ s$. Avec ce type de cache, il est facile d'ajouter ou de retrouver une données. Cependant, si plusieurs blocs mémoires correpondant à la même ligne de cache sont fréquemment utilisés, il faudra sans cesse supprimer et ajouter des données dans le cache.

\subsubsection{\emph{Fully associative}}
Un cache est en correspondance associative si chaque bloc mémoire peut être mis dans n'importe quelle ligne du cache. Il n'y a alors qu'un seul \textit{set}. L'inconvénient précédent n'est plus existant, cependant il devient beaucoup plus compliqué de rechercher une donnée dans le cache. L'ensemble des étiquettes doit en effet être parcouru.

\subsubsection{\emph{$k$-ways associative}}
Les deux cas présentés précedemment présentent chacun des inconvénients. Généralement, un cache profite des avantages des deux visions en faisant un compromis. Dans le cas de la correspondance associative par ensemble, chaque \textit{set} possède un nombre $k$ de lignes tel que $p=k \times s$. La fonction de correspondance est telle que $i = j\ modulo\ s$. De cette manière, un bloc mémoire peut se trouver dans un ensemble de $k$ lignes. Il est donc possible d'avoir plusieurs blocs mémoires correspondant au même ensemble et l'algorithme de recherche est plus efficace que dans le cas de la correspondance associative.

\begin{figure}[H]
\begin{center}
   \includegraphics[scale=0.60]{images/associative.png}
   \caption{\label{img:associative} Fonction de correspondance - www.root.cz}
\end{center}
\end{figure}

%==> Préciser source de l'image\\

\subsection{Politiques de remplacement}
\label{remplacement}
 Lorsque l'on souhaite ajouter une ligne dans un \textit{set} plein, il faut au préalable évincer une ligne de ce \textit{set}. Pour cela, il existe différentes méthodes :

\paragraph{FIFO.} La première solution consiste à supprimer la ligne la plus ancienne du cache. Cela correspond à la formule: ``First In, First Out''.

\paragraph{LFU.} Une autre solution consiste à supprimer la ligne qui a été le moins utilisée: ``Least Frequently Used''. Pour cela, chaque ligne de cache possède un compteur qui sera incrémenté à chaque utilisation de la ligne. 

\paragraph{LRU.} La dernière solution, généralement utilisée, consiste à favoriser le principe de localité temporelle en supprimant la ligne du \textit{set} qui a la plus ancienne date d'utilisation: ``Least Recently Used''.

\section{Gestion de la cohérence}
Dans les systèmes actuels, un processeur n'est plus composé d'un unique c{\oe}ur, mais de plusieurs. Chaque c{\oe}ur possède généralement deux caches qui lui sont propres : le L1i pour stocker les instructions et le L1d pour stocker les données. Les caches de niveau supérieur sont quant  eux partagés par un nombre variable d'autres c{\oe}urs. Alors que les interactions entre ces deux premiers caches sont minimes -- il est difficile de mélanger instructions et données --, la cohérence entre les différents L1d est en revanche un problème de taille.

\subsection{Présentation du problème}
Les caches sont utilisés à chaque \textit{load}/\textit{store}. S'il est possible que plusieurs caches possèdent la même donnée et la lisent en même temps, il est primordial qu'une donnée ne puisse pas être modifiée simultanément dans deux caches. Pour cela, des techniques hardware sont mises en place afin de définir qui a la priorité si plusieurs c{\oe}urs veulent modifier une même donnée. \\

\begin{figure}[H]
\begin{center}
   \includegraphics[scale=0.35]{images/learn_mesi_2.png}
   \caption{\label{img:mesi} Utilité d'un protocole de cohérence}
\end{center}
\end{figure}

Par ailleurs, un protocole de cohérence est mis en {\oe}uvre à chaque \textit{load}/\textit{store} afin que les différents caches soient informés des modifications les concernant et que la consistance du système soit assurée. Ce protocole est propre à un niveau de cache. Dans un cas classique, il y aura un protocole de cohérence entre les L1 et entre les L2.

\subsection{Protocoles de cohérence}
\label{coherence}
 Nous étudierons uniquement le protocole de type MSI et ses dérivés : MESI, MOSI, MOSIF et MOESI. Chaque ligne de cache possède un état qui permet de gérer la cohérence. Les différents états sont: \\
\begin{itemize}
\item M: Une ligne est dans l'état modifié si c'est la seule copie valide dans l'ensemble des caches du niveau. Dans ce cas, si la ligne est evincée du cache, elle doit être recopiée en mémoire, via un \emph{write back}. \\
\item S: Une ligne est dans l'état partagé (\textit{shared}) si elle est valide et qu'elle n'a pas été modifiée (hormis dans le cas de MOESI). Dans ce cas, plusieurs caches peuvent possèder la ligne. \\
\item I: L'état invalide est utilisé pour une ligne qui n'est pas valide. Le contenu de la ligne n'étant pas valable, il ne faut pas l'utiliser. \\
\item E: Une ligne est dite exclusive si elle est, dans le niveau, la seule copie valide. Une ligne dans cet état n'a pas été modifiée et les données de la ligne sont donc identiques à celles de la mémoire principale. \\
\item O: L'état \textit{owned} est utilisé pour un cache qui possède une donnée modifiée par rapport à la mémoire principale. Plusieurs caches peuvent possèder la même donnée, ils seront alors dans l'état S. \\
\item F: L'état \textit{forward} permet de définir, dans le cas du \emph{snooping}, quel cache doit répondre à la réquête de \emph{snooping}. De cette manière, un seul des caches possèdant la donnée répond et cela évite d'utiliser inutilement de la bande passante.\\
\end{itemize}

L'état M est utilisé lorsque la donnée a été modifiée par un c{\oe}ur. Il existe deux cas de propagation des modifications. Dans le choix de la politique \textit{write-through}, la donnée est directement recopiée dans la mémoire principale pour éviter de futurs problèmes de cohérence. Dans le cas de la politique \textit{write-back}, la donnée est modifiée uniquement dans le cache. Les autres caches et la mémoire principale peuvent savoir que la donnée a été modifiée, en revanche ils n'ont pas la dernière copie valide. Ils peuvent l'obtenir lors des \textit{Write-Back}: lorsque la donnée est evincée du cache ou lorsqu'elle est demandée à un plus haut niveau pour des soucis de cohérence.\\

Le protocole de cohérence le plus utilisé est MESI. Voici certaines caractéristiques de ce protocole : \\
\begin{itemize}
\item Lorsqu'un cache charge une donnée, il demande si un autre cache possède la donnée. Si un cache possède la donnée dans l'état M, alors ce cache fait un \emph{write-back} et invalide sa donnée. Si un autre cache possède la donnée en état E, la donnée passe dans l'état S pour les deux caches. Si plusieurs (ou un seul) caches possèdent la donnée dans l'état S, la donnée est ajoutée dans l'état S.
\item Si un cache possède une donnée dans l'état M ou E et fait un nouveau \textit{store} sur la ligne de cache, son état devient M et aucune information n'est transmise aux autres caches.
\item Si un cache fait un \textit{store} sur une donnée qui est dans l'état S, la même donnée est invalidée dans les autres caches du même niveau. \\
\end{itemize}

Le protocole est souvent représenté sous la forme d'un automate, dont les états possibles sont ceux des données : \\

\begin{figure}[H]
\begin{center}
   \includegraphics[scale=0.45]{images/mesi.png}
   \caption{\label{img:mesi_aut} Automate du protocole MESI - unisim.org}
\end{center}
\end{figure}
Par rapport à l'automate, voici les différentes significations: \\
\begin{itemize}
\item Rd: le cache courant effectue une lecture,
\item Wr: le cache courant effectue une écriture,
\item Pr: modification en privé en partant d'un certain état,
\item Bus: envoie d'un message à l'ensemble des caches en parallèle d'une modification privée. \\
\end{itemize}

Pour l'état O, la différence est que les interactions avec la mémoire principale sont moins importantes. En effet, lorsqu'un cache qui n'a pas la donnée fait un \textit{load}, le cache possédant la donnée dans l'état O peut lui donner sans faire de \textit{write-Back}. Cela permet de garder plus longtemps la donnée modifiée.

%\newpage
\section{Fonctionnement global}
La cohérence expliquée jusqu'à présent correspondait à celle d'un niveau de caches. Mais il peut arriver d'autres problèmes de gestion des données entre ces différents niveaux car les niveaux de caches peuvent être inclusifs, exclusifs ou non-inclusifs. Par ailleurs, les particularités du \emph{snooping}, qui consiste à demander via un bus une donnée aux autres caches du même niveau, seront aussi explicitées. \\

Le cas inclusif est plutôt orienté \textsf{Intel} alors que le cas exclusif est plutôt mis en place par \textsf{AMD}. De son côté \textsf{ARM} oscille entre les deux solutions proposées.

\label{inclusivite}
\subsection{Caches inclusifs}
Le fonctionnement d'un cache inclusif est relativement simple. En cas de \textit{hit}, la donnée est propagée au cache (ou au processeur) qui a préalablement demandé la donnée. En cas de \textit{miss}, une demande de donnée est effectuée au niveau du dessus. La donnée est ensuite propagée à la mémoire d'en dessous. Dans le cas d'une hiérarchie avec uniquement des caches inclusifs, les messages utilisés pour demander/donner les données se font uniquement de façon verticale dans la hiérarchie. Il y a parallèlement, des messages destinés à garantir la consistance du système qui sont gérés différemment et qui influent sur la totalité de la hiérarchie. Voici un résumé des messages envoyés dans le cas d'une demande de donnée, en ne tenant pas compte de la gestion de la cohérence : \\

\begin{figure}[H]
\begin{center}
   \includegraphics[scale=0.7]{images/inclusifs.png}
   \caption{\label{img:cache_inclusifs} Caches de niveau 2 et 3 inclusifs}
\end{center}
\end{figure}

Dans le schéma étudié, un \emph{load} de la donnée $A$ est effectué sur le quatrième cache, puis sur le premier cache. Les données contenues dans les différents caches sont donc appelées $Ai$ avec $i$ l'étape $1$ ou $2$.

\subsection{Caches exclusifs}
Cependant, le fait qu'un cache soit inclusif pose un problème : les données sont dupliquées dans la hiérarchie. De ce fait, la taille réellement utilisable de l'ensemble des caches est en fait inférieure à la taille théorique. Les caches exclusifs ont justement pour principal objectif de limiter cette duplication des données. Par exemple, si un L2 est exclusif, les L1 en dessous ne pourront pas contenir une donnée déjà présente dans le L2, et inversement. Concrètement, il y a quatre cas de figure : \\

%==>un joli schéma\\
\begin{itemize}
\item Si un L1 fait un \textit{hit}, il transmet la donnée au c{\oe}ur associé
\item Si un L2 fait un \textit{hit}, il donne la donnée au L1 qui l'a demandé puis la supprime.
\item Si un L1 fait un \textit{miss}, il effectue une demande de donnée au L2 associé. Il gardera la donnée une fois qu'elle lui sera transmise.
\item Si un L2 fait un \textit{miss}, il effectue une demande de donnée au L3 associé, ou directement à la mémoire principale si il n'y a pas de L3. Une fois que la donnée lui sera transmise, il la propagera au L1 sans la conserver. \\
\end{itemize}

Il en va de même pour les interactions entre les L2 et les L3, voire d'autres niveaux dans le cas d'une architecture plus grande. Par ailleurs, lorsque le dernier niveau fait un \textit{miss}, il faut parcourir l'ensemble de la hiérarchie avant de rechercher directement la donnée dans la mémoire principale. Ce phénomène explique à lui seul le fait que la gestion de la recherche et de la cohérence soit plus délicat que dans le cas d'une hiérachie avec le dernier niveau inclusif.

\subsection{Caches non-inclusifs}
Il existe un autre cas de figure, qui peut être considéré comme une variante du cas exclusif. Lorsqu'un L1 demande une donnée et que le L2 ne l'a pas, il se contentera de propager la donnée sans la conserver en mémoire. Cependant, si il avait préalablement la donnée, il la propagera au L1 sans pour autant la supprimer. Ce cas de figure peut être utilisé pour un L2 avec un L3 inclusif. De cette manière, la hiérarchie permet en moyenne de stocker plus de blocs mémoires sans pour autant compliquer la gestion de la cohérence. \\

Un autre cas de figure est un cache non-inclusif qui fonctionne avec plus de souplesse qu'un cache inclusif. Les données sont toujours laissées au passage, par contre l'inclusivité n'est pas formelle: il peut perdurer dans un L1 une donnée qui a été evincée d'un L2 par exemple.

\subsection{Ajout du \emph{snooping}}
\label{snooping}
 Il est possible que des caches d'un même niveau soient reliés ensemble par un bus. La méthode de recherche sur ce bus est alors appelée \emph{snooping}. Prenons l'exemple d'une hiérarchie à trois niveaux où les L2 seraient reliés par un bus. Lorsqu'un L1 fait un \textit{miss}, il demande au L2 associé de lui fournir la donnée. En cas de nouveau \textit{miss}, la demande est envoyée en broadcast sur le bus afin de récupérer la donnée sans passer par le L3. Si aucun cache ne possède la donnée, la demande est alors transmise au L3.

\section{Comportements spéciaux}
Les politiques de remplacement, de cohérence et les différents types de caches offrent un ensemble de modularités dans la gestion d'une hiérarchie de caches. Cependant, d'autres possibilités existent afin d'améliorer les performances de l'ensemble.

\subsection{Utilisation d'un \textit{victim cache}}
Généralement, des caches à $k$ voies associatifs sont utilisés, car la fonction de recherche d'une donnée est peu coûteuse et plusieurs blocs mémoire correspondant à un même \textit{set} peuvent être stockés dans un cache. Dans la plupart des cas, le nombre de voies varie entre $2$ et $12$. \\

Dans certains cas, il est possible que le nombre de voies ne soit pas suffisant car un ensemble de blocs mémoire correspondant au même \textit{set} est très utilisé. Cela est notamment vrai pour les L1 qui possèdent généralement peu de voies. Pour pallier ce problème et limiter les échanges de données entre les L1 et les L2, il est possible d'utiliser un buffer pour stocker les victimes d'éviction. Ce buffer est de petite taille, par exemple $8$ fois la taille d'une ligne de cache. De cette manière, lorsqu'un donnée est supprimée du L1, elle est placée dans le buffer correspondant. Lorsque le L1 fait un \textit{miss}, il commence par regarder dans le buffer (le coût est alors faible). Les données arrivant dans le L2 sont alors celles evincées du buffer.

\subsection{Suivi des données}
 Il existe différentes manières de localiser les données dans une hiérarchie mémoire, il ne s'agit pas ici de la recherche au sein d'un cache, cela est géré par la table d'étiquettes, mais bien entre les caches. Deux cas fréquents se posent, ils sont réglés la plupart du temps par un \textit{tracking} : savoir dans quels caches se trouvent les données.

\subsubsection{Supprimer les données}
Lorsqu'un cache L1 utilise une donnée qu'il possède déjà dans son cache, les flags utilisés par les politiques de remplacement sont mis à jour afin de supprimer les bonnes données lorsqu'un \textit{set} sera plein. Cependant, les caches de plus haut niveau (L2, L3) ne sont pas mis au courant que la donnée a été utilisée et ils ne changent donc pas les flags de remplacement. Ainsi, lorsque le L2 aura un \textit{set} plein, il se peut qu'il supprime une donnée très utilisée dans un L1. \\

Dans le cas d'un cache inclusif, il faut donc, à la suppression d'une donnée, invalider la donnée dans les caches en dessous afin de conserver le caractère inclusif. Pour limiter ce problème, les L2 et les L3 peuvent suivre les données, c'est-à-dire posséder une table indiquant quels caches de niveau plus bas ont également la donnée. De cette manière, les politiques de remplacement peuvent être adaptées en définissant des priorités: par exemple, éviter de supprimer les données qui sont contenues dans beaucoup de caches.

\subsubsection{Savoir qui a les données}
Dans le cas d'un cache de dernier niveau et exclusif, s'il se produit un \textit{miss}, il faut parcourir l'ensemble de la hiérarchie avant d'éventuellement faire appel à la mémoire principale. Cela pose des problèmes, tant au niveau de la recherche des données que de la gestion de la consistance du système. Pour pallier ce problème, le cache de dernier niveau peut être lié à une table d'étiquettes, permettant de recenser les données contenues dans chaque cache. Cela évite d'avoir à parcourir l'ensemble de la hiérarchie mais il y a un coût en terme de mémoire. Cependant, ce coup est faible (les données ne sont pas stockées) et bien inférieur au coût mémoire de la duplication dans le cas d'un cache de dernier niveau inclusif.

\newpage
\chapter{Presentation du simulateur de caches : \emph{Cassis}}


===> D'autres trucs sur l'implém, par exemple préciser ce que ne fait pas le simulateur (calcul de bande passante, prise en compte du prefetching) ???


\section{Cadre du simulateur}

Un simulateur est un programme qui va reproduire le fonctionnement d'un système afin de pouvoir en étudier certaine caractéristiques de manière moins contraignante que sur le système réel. Dans le cas du programme \textsc{Cassis}, c'est l'utilisation des caches par un programme donné qui est simulée.

\subsection{Origine du projet}

La mémoire est un facteur de ralentissement des programmes très important de nos jours. Une mauvaise gestion de la mémoire peut être désastreuse au niveau des performances, et pourtant, il n'existe pas de moyen efficace et précis de connaître l'utilisation de la mémoire au niveau des caches. Le but de \textsc{Cassis} est donc d'étudier l'utilisation précise des caches par un programme, afin d'aider l'utilisateur à tester les performances du programmes, ainsi que les analyser pour les améliorer.

\paragraph{}
Le besoin d'un tel simulateur est avant tout dû au manque d'informations des caches. Deux phénomènes se produisent du côté des constructeurs de processeurs. D'une part ils doivent les optimiser afin d'être concurrents, et il devient alors difficile pour eux de rajouter des fonctionnalités dédiées aux développeurs afin d'améliorer leur code, par exemple des compteurs exacts du nombre de \emph{hits}. D'autre part il n'est pas dans leur intérêt de révéler le fonctionnement de leurs produits, pour des raisons économiques évidentes. Par exemple, le \emph{prefetching} permet de charger des lignes de caches en avance en fonction des accès mémoires courants du programme et cette fonctionnalité n'est pas implémentée dans le simulateur car son fonctionnement n'est pas totalement divulgué.

\paragraph{}
L'intérêt d'un tel simulateur est donc de pouvoir tester des programmes en produisant certaines statistiques, mais aussi de pouvoir tester l'exécution sur des architectures qui n'existent pas afin de connaître la pertinence d'une nouvelle politique de cohérence par exemple.

\subsection{Données simulées : analyses possibles des résultats}

Différentes statistiques sont relevées par le simulateur, les principales sont les \emph{misses}, les \emph{hits} et les \emph{write-backs}. Il est aussi possible de connaître le nombre de messages envoyés sur le bus par un cache afin d'effectuer la cohérence, ainsi que le nombre de données qui ont été invalidées à cause de la cohérence ou à cause de l'inclusivité, et enfin comment la donnée a été atteinte (qui l'a transmis au cache s'il ne l'avait pas : son parent, ou un cache de même niveau). Une sortie classique serait donc :

\begin{lstlisting}
L3  basics   (misses:   6, hits:   0, writes_back:   0)
L2  basics   (misses:   6, hits:   0, writes_back:   0)
L1  basics   (misses:   4, hits:  44, writes_back:   1)
L1  basics   (misses:   6, hits:  42, writes_back:   2)
L2  basics   (misses:   0, hits:   0, writes_back:   0)
L1  basics   (misses:   0, hits:   0, writes_back:   0)
L1  basics   (misses:   0, hits:   0, writes_back:   0)
\end{lstlisting}

\paragraph{}
Ces statistiques ont été choisie afin que l'utilisateur puisse constater certains problèmes particuliers. En effet un nombre trop élevé de \emph{write-backs} et/ou de \emph{coherence evinctions} révèlent que les c\oe urs modifient les mêmes données et que le parallélisme n'est donc pas bien réalisé. Mais les statistiques ne sont pas les seuls paramètres d'analyse, la sélection des données à analyser à aussi un rôle important. Le simulateur permet de suivre les statistiques d'une ou plusieurs instructions données, et ainsi de déterminer quelles instructions causent des ralentissements à l'exécution, i.e. un trop grand nombre de \emph{misses}.

\subsection{Outils à disposition de l'utilisateur}

Le simulateur s'appuie sur des outils préexistants, pour la génération des traces et de l'architecture. Les outils listés ci-dessous sont ceux utilisés par l'équipe de projet. Si d'autres outils offrent des fonctionnalités similaires, ils peuvent être utilisés en remplacement tant que les fichiers d'entrée du programme sont dans le bon format.

\paragraph{HWLOC} est un logiciel qui permet de récupérer l'architecture des caches d'une machine. Il permet de générer un fichier xml que nous enrichissons. La paramétrisation de l'architecture des caches ne prend pas en compte les caches d'instructions, mais seulement les caches de données. L'utilisation de ce fichier est détaillé dans la section \ref{param_xml}.

\paragraph{MAQAO} (Modular Assembly Quality Analyzer and Optimizer) est un outil d'analyse et d'optimisation de programmes. Une seule fonctionnalité de \textsf{MAQAO} est utilisée, celle qui permet d'instrumenter un programme binaire afin de récupérer les opérations faites lors de l'exécution. Le paramétrage de \textsf{MAQAO} est effectué grâce à un fichier lua pour générer des traces de la forme voulue.

\paragraph{}
L'utilisateur peut aussi vouloir comparer les résultats de la simulation. Bien qu'il n'existe par actuellement de tel simulateur pour faire la comparaison, il existe des compteurs hardware --par exemple \textsf{PAPI} -- qui permettent de connaître certaines statistiques sur les caches. Ces statistiques sont celles desquelles il faudrait le plus se rapprocher, bien qu'elles ne soient pas non plus exactes.


\section{Déroulement de la simulation}

La simulation consiste à rejouer un certain nombre d'instructions, et de comptabiliser certaines métriques à destination de l'utilisateur. Comme il s'agit d'une simulation, tout ne se passe pas exactement comme dans le cas réel. Il est donc nécessaire d'expliciter les similitudes comme les différences afin que l'utilisateur ne soit pas surpris à la fois par les résultats, et à la fois par les méthodes de calculs si son objectif est de modifier le code.

\subsection{Traitement d'une instruction : load/store}

La simulation, même d'un programme parallèle, est une séquence d'instructions effecutant des modifications (lecture ou écriture) sur les caches. Les deux premières possibilités pour une instruction sont soit une lecture de donnée, soit une écriture. Une fois cela déterminé, il faut savoir où se trouve la donnée, elle peut déjà être dans le cache, ou dans un de ses voisins, ou dans un de ses parents. La donnée est alors rappatriée en prévenant les autres caches de l'action en cours. 

\paragraph{}
Tous les caches concernés appliquent alors la politique de cohérence : une action a été faite sur la donnée, elle a été lue ou modifiée, par un autre cache ou par soi-même. L'automate change alors l'état de la ligne suivant la nature de la transition. Il faut aussi penser à gérer le cas ou le cache est plein : le rappatriement d'une donnée donne alors lieu à la suppression d'une autre, il faut aussi propager cette information, et veiller à la cohérence de la donnée dans le reste de l'architecture.

\paragraph{}
L'ordre des tests est le suivant : 
\begin{enumerate}
  \item{il s'agit d'un \emph{load} ou d'un \emph{store}}
  \item{la donnée est présente dans le cache ou non (récursif)
    \begin{enumerate}
      \item{si non, l'ajouter}
      \item{appel de la procédure de suppression en cas de cache plein}
      \item{prévenir le cache supérieur que la donnée est accédée}
      \item{revenir au point 2}
  \end{enumerate}}
  \item{modifier les statistiques d'utilisation de la ligne}
\end{enumerate}

\subsection{Rapatriement prédictif d'une ligne}

Les étapes précédentes de l'algorithme montrent que la ligne est ajoutée avant même que les caches de niveau supérieur aient confirmé qu'ils l'ont ou non. Contrairement au cas réel, nous considérons donc que nous avons la donnée dans tous les cas, et nous propageons uniquement la demande aux niveaux supérieurs. Cela augement la rapidité car il n'y a pas d'aller/retour de messages entre les niveaux de caches, ceux-ci ne font \emph{a priori} que monter vers les caches supérieurs.

\paragraph{}
Cette particularité est à prendre en compte dans l'écriture des politiques de cohérences. En effet certaines politiques se basent sur la présence ou non de la même ligne dans le niveau du cache demandant la donnée. Il faut alors être conscient que la donnée est déjà dans le niveau lorsque les voisins sont prévenus.

\paragraph{}
En résumé il faut considérer que le cache effectuant la requête de ligne l'ajoute, prévient ses voisins, puis prévient le niveau supérieur. La cohérence entre les niveaux est toujours assurée, mais celle sur le niveau-même n'est donc pas exactement la même que dans le cas réel où les voisins sont informés avant l'ajout effectif de la ligne.


\subsection{Mise à jour des lignes}

Lorsque les lignes sont accédées il faut mettre à jour leurs statistiques d'utilisation suivant la politique de remplacement du cache associé. Cette mise à jour intervient dès que la ligne a été ajoutée si elle n'était pas déjà présente. Et ce avant même d'avoir prévenu les caches parents, mais après avoir prévenu les caches voisins.

\paragraph{}
Le type de cache, inclusif ou exclusif ou autre, influe ici sur la mise à jour des lignes. En effet dans le cas d'un cache orienté inclusif, celui-ci contiendra effectivement la donnée rappatriée et il mettra à jour son utilisation. Mais remarquons que dans le cas d'un cache exclusif, celui-ci ne fait que faire transiter la donnée vers le cache inférieur, donc il ne faut pas augmenter les statistiques d'utilisation de cette ligne qui n'est pas présente.

\subsection{Problème d'ajout de ligne dans un cache plein}

Lorsqu'un cache nécessite une donnée alors qu'il est plein, il en supprime une autre. Mais dans le cas d'une architecture totalement inclusive, cela pose un problème lié au faite que si l'on supprime la donnée d'un niveau inclusif, il faut la supprimer dans tous les niveaux inférieurs.

\paragraph{}
Le cas est le suivant : une donnée A est accédée par un L1. Pour ce faire le L1, éjecte la donnée B. Or le L2 est inclusif, donc il doit stocker la donnée B, mais pour ce faire, étant plein aussi, il doit supprimer la donnée A. Par conséquent l'inclusivité pour A n'est plus vraie, elle est maintenant dans le L1 mais plus dans le L2. 

\paragraph{}
Le rapatriement prédictif pose donc ici un problème qui ne se pose pas dans le cas réel. En effet dans le cas réel, le L2 ayant reçu la demande de la part du L1 pour la donnée A, évince la donnée B, prévient le L1 que cette donnée a été supprimée, puis seulement après lui transmet la donnée A. Le L1 a donc au moins une place disponible pour recevoir cette donnée et n'en suprrime pas d'autres. 

\paragraph{}
Pour pallier ce problème, une donnée supplémentaire a été ajouté aux différentes fonctions : il s'agit de la ligne en cours d'ajout, qui ne doit pas être supprimée par les caches de niveaux supérieurs.

\subsection{\'Etapes de validation}
\begin{frame}
  \begin{block}{Pr\'esentation}
    \begin{itemize}
      \item Tests unitaires
      \item Validation durant le programme
      \item Comparaison avec des outils existants
      \item Benchmarks
    \end{itemize}
  \end{block}
\end{frame}

\begin{frame}
\begin{figure}[H]
   \begin{minipage}[l]{.46\textwidth}
     \includegraphics[scale=0.25]{images/stats_L1.png}
   \end{minipage} \hfill
   \begin{minipage}[r]{.46\textwidth}
     \includegraphics[scale=0.25]{images/stats_L2.png}
   \end{minipage}
\end{figure}

\begin{figure}[t!]
  \includegraphics[width=.4\textwidth]{images/stats_L3.png}
\end{figure}
\end{frame}

\subsection{Limites \`a  propos de la simulation des caches}
\begin{frame}
  \begin{block}{Limites usuelles}
    \begin{itemize}
      \item \textsf{prefetching}
      \item synchronisations
      \item changements de contextes 
    \end{itemize}
  \end{block}
\end{frame}


\section*{Conclusion}
\begin{frame}
  \begin{block}{Objectifs atteints}
    \begin{itemize}
      \item Cahier des charges respect\'e
      \item Param\'etrisation compl\`ete
    \end{itemize}
  \end{block}

  \begin{block}{\'Evolution possible}
    \begin{itemize}
      \item Complage avec un simulateur \textsf{on-line}?
      \item Utilisation de benchmarks pour calibrer les r\'esultats?
    \end{itemize}
  \end{block}
\end{frame}


\newpage
\chapter{Paramétrisation du simulateur}
Un simulateur présente l'intérêt de se dégager de certaines contraintes imposées aux expériences sur des systèmes réels. La simulation ne serait donc que peu utile s'il existait trop de contraintes. La forte paramétrisation du simulateur \textsf{Cassis} était donc un objectif impératif, introduit dès la phase de spécification des besoins. Il convient d'avoir la possibilité de simuler des programmes sur des architectures et des politiques de cohérence et d'entrelacement descriptibles facilement. Des langages spécifiques ont été utilisés pour chacun des aspects, afin soit de permettre une modification rapide des paramètres de simulation sans recompilation, soit de permettre une modification dans un langage adapté au problème.

\section{Paramétrisation de l'architecture}

\label{param_xml}
L'architecture à simuler peut être générée à partir de l'architecture réelle de l'utilisateur au moyen d'un fichier XML créé par le logiciel \textsf{HWLOC}. Cependant l'utilisateur peut utiliser un fichier de paramétrisation spécifique à notre simulateur qui lui permet d'accéder à l'intégralité des paramètres d'architecture pris en compte.

\subsection{Entrée XML HWLOC}

\textsf{HWLOC} est un logiciel libre sous licence BSD-2. Il permet de générer un fichier XML qui décrit l'architecture de la machine utilisée (commande \verb?lstopo --of xml?). Il décrit notamment la structure arborescente des caches, et donne des informations essentielles pour chaque cache, comme sa taille, la taille de ses lignes et son associativité. 

\paragraph{}
Si l'utilisateur choisit un tel fichier en entrée comme décrivant son architecture, ce dernier sera parsé par une feuille \textit{xslt} en un fichier de configuration de l'architecture personnalisé, comme décrit dans la section \ref{config}. Les paramètres non fournis par le fichier généré par \textsf{HWLOC} prendront des valeurs par défaut, proches de celles des architectures \textsf{Intel} moderne (voir l'exemple en annexe \ref{manarchi}). Cette organisation est décrite par un schéma figure \ref{img:archi}. Notons que notre simulateur ne prend pas en compte les caches de niveau 1 dédiés aux instructions (L1i), qui sont décrits par \textsf{HWLOC} mais ne seront pas présent dans le fichier personnalisé.

\begin{figure}[!h]
\begin{center}
   \includegraphics[width=0.7\textwidth]{images/schema_archi.png}
   \caption{\label{img:archi} Fichiers de paramétrisation de l'architecture}
\end{center}
\end{figure}

\subsection{Fichier de configuration personnalisé}
\label{config}
Le fichier de configuration de l'architecture dédié à notre simulateur comprend tous les paramètres d'achitecture utilisables. Une fois généré à partir d'un fichier \textsf{HWLOC}, il est possible de l'utiliser directement en entrée du simulateur, après avoir été modifié à la convenance de l'utilisateur. La documentation précise ainsi qu'un exemple de fichier \textit{.cassis.xml} est disponible dans une page de manuel, donnée en annexe \ref{manarchi} page \pageref{manarchi}.

\paragraph{}
Il s'agit d'un fichier XML qui contient 3 balises : \textbf{Architecture}, \textbf{Level} et \textbf{Cache}.

\paragraph{Architecture} donne les infomations générales sur l'architecture, telles que le son nom, le modèle du microprocesseur et le nombre de niveaux de cache.

\paragraph{Level} décrit les informations relatives à un niveau de cache en particulier. Il possède notamment les attributs suivants :
\begin{itemize}
  \item \textbf{coherence\_protocol} : MESI, MSI, MOSI, MESIF, ou MOESI (cf. \ref{coherence} page \pageref{coherence}), qui sont les protocoles implémentés pour le simulateur. Le dernier niveau ne possède pas de protocole de cohérence, car il est le seul cache dans son niveau.
  \item \textbf{type} : l'inclusivité des caches du niveau (cf. \ref{inclusivite} page \pageref{inclusivite}). Ce paramètre n'a pas de sens pour les L1.
  \item \textbf{snooping} : y a-t-il du \emph{snooping} à ce niveau ? (cf. \ref{snooping} page \pageref{snooping}). Le cache du dernier niveau ne peut pas faire de \emph{snooping}.
  \item \textbf{directory\_manager} : les caches du niveau possèdent-il un \emph{directory manager} ? Ce paramètre n'est pas pris en compte pour les L1.
\end{itemize}

\paragraph{Cache} concerne les informations spécifiques à un cache en particulier. Il donne notamment la taille en octets du caches, la taille d'une ligne de cache, l'associativité du cache, et enfin le protocole de cohérence du cache(FIFO, LRU ou LFU, cf. \ref{remplacement} page \pageref{remplacement}).

\paragraph{}
Il est ainsi possible de simuler un bon nombre d'architectures, même si certaines n'ont pas de sens. Le simulateur effectue une vérification sur l'achitecture avant la simulation, afin d'écarter certaines architectures qui ont de grandes chances de faire planter le programme. Nous avons néanmoins laissé à l'utilisateur la possibilité de passer outre ces vérifications.



\section{Entrelacement des threads}

Les traces simulées sont générées par \textsf{MAQAO}, elles représentent une suite linéaire d'instructions. Elles ne reflètent donc pas le comportement réel de l'exécution parallèle du programme où les instructions sont réparties entre les différents c\oe ur du processeur. 

\subsection{Nécessité de l'entrelacement}

Sur une architecture multi-coeur plusieurs threads sont exécutés à la fois. Les threads exécutent donc des instructions en parallèle permettant ainsi d'augmenter la vitesse de calcul. Le problème est qu'il est possible de savoir sur quel c\oe ur une instruction a été exécutée, mais pas sa position dans l'exécution global du programme. En effet un c\oe ur a pu recevoir un appel système et temporairement exécuter d'autres opérations que celles du programme simulé, l'ordre d'exécution des instructions sur tous les c\oe ur ne peut donc pas a priori être connu. 

\paragraph{}
Un modèle de base a donc été implémenté afin de configurer l'ordre d'exécution des traces, appelé entrelacement des threads. Celui choisi pour le simulateur fonctionne de la manière suivante : un certain nombre d'instructions sont exécutées sur un thread, une fois terminées ce sont autant d'instructions qui sont exécutées sur le c\oe ur suivant (modulo le nombre de c\oe urs, car c'est un cycle). Cela revient à séquencialiser le programme en entrelaçant les instructions sur les différents threads.

\subsection{Entrelacement bloc par bloc}

Le choix d'entrelacement qui a été fait permet donc de changer de coeur sur lequel il faut exécuter les instructions de manière cyclique. La technologie utilisée est un script LUA qui retourne le prochain coeur sur lequel le simulateur doit exécuter les instructions. Ce script permet de fixer un nombre d'instructions à exécuter avant de changer de coeur, pour changer de coeur à chaque instructions il faut donc réduire la taille du bloc d'instructions à exécuter à une seule instruction.\\

===============> Joli schéma de l'exécution bloc par bloc


\paragraph{}
Le modèle d'entrelacement des threads étant placé dans un fichier LUA, celui-ci est modifiable sans nouvelle compilation. Dans le cas où les traces n'ont pas toutes le même nombre d'instructions, il n'est toutefois pas possible depuis le script de savoir si une trace est terminée alors que les autres non. Si le c\oe ur renvoyé par le script n'a plus d'instruction à lire sur la trace associée, rien ne se produira, et le script renverra le c\oe ur suivant. 


\begin{frame}{Nécessité de la cohérence}
    \includegraphics[scale=.3]{images/learn_mesi_2.png}  
\end{frame}

\begin{frame}{Solution MESI}
  \begin{block}{Cohérence = automate}
    Les lignes de caches respectent certaines caractéristiques d'un automate : ont un nombre fini d'états, et passent de l'un à l'autre suivant des actions extérieurs (i.e. étiquettes/transitions).\\
    Pour \textsf{Cassis}, usage de \textsf{SMC} afin de représenter les automates de cohérences.
  \end{block}
  
  \begin{block}{Les états :}
    \begin{itemize}
    \item{I (donnée invalide),}
    \item{M (donnée modifiée),}
    \item{S (donnée partagée),}
    \item{E (donnée unique).}
    \end{itemize}
  \end{block}
\end{frame}

\begin{frame}{Automate MESI : les transitions}
    \includegraphics[scale=.3]{images/MESI_simple.png}
\end{frame}

\begin{frame}{Autres protocoles}
  \begin{block}{Protocoles implémentés}
    Les autres protocoles courrants sont :
    \begin{itemize}
    \item{MSI (le plus basique),}
    \item{MOSI/MESIF (alternative presqu'équivalente à MESI),}
    \item{MOESI (le plus complexe).}
    \end{itemize}
  \end{block}
  
  \begin{block}{Avantages/Inconvénients}
    Deux critères importants pour le choix d'un protocole :
    \begin{itemize}
    \item{le nombre de \emph{write-back},}
    \item{l'utilisation de la bande-passante.}
    \end{itemize}
  \end{block}
\end{frame}



\newpage
\appendix
\chapter{Pages de man du programme}
\section{Manuel de paramétrisation de l'architecture}
\label{manarchi}
\begin{lstlisting}
FILE DESCRIPTION
       <Architecture>: decribes global information concerning the architecture

            * name: name of the architecture
            * CPU_name: name of the CPU
            * number_levels: number of levels if the cache hierarchy

       <Level>: parameters shared with a cache level

            * depth: depth of the level
            * coherence_protocol (not for last level): MSI or MESI
            * type (not for L1): inclusive, exclusive, nieo (Not Inclusive Exclusive Oriented) or niio (Not Inclusive Inclusive Oriented).
            * snooping (not for last level): true or false
            * directory_manager (note for L1): true or false

       <Cache>: parameters specific to a cache

            * depth: depth of the cache
            * cache_size: size of the cache (in bytes)
            * cache_linesize: size of a line in the cache (in bytes)
            * cache_associativity: associativity of the cache
            * replacement_protocol: LRU, LFU or FIFO

EXAMPLE
       <?xml version="1.0"?>
       <Architecture name="x86_64" CPU_name="Intel(R) Core(TM) i5-3340M CPU @ 2.70GHz (modified)" number_levels="3">
         <Level depth="3" type="inclusive" directory_manager="false"/>
         <Level depth="2" coherence_protocol="MESI" type="inclusive" snooping="true" directory_manager="false"/>
         <Level depth="1" coherence_protocol="MESI" snooping="true"/>
         <Cache depth="3" cache_size="3145728" cache_linesize="64" cache_associativity="12" replacement_protocol="LRU">
           <Cache depth="2" cache_size="262144" cache_linesize="64" cache_associativity="8" replacement_protocol="LRU">
             <Cache depth="1" cache_size="32768" cache_linesize="64" cache_associativity="8" replacement_protocol="LRU"/>
             <Cache depth="1" cache_size="32768" cache_linesize="64" cache_associativity="8" replacement_protocol="LRU"/>
           </Cache>
           <Cache depth="2" cache_size="262144" cache_linesize="64" cache_associativity="8" replacement_protocol="LRU">
             <Cache depth="1" cache_size="32768" cache_linesize="64" cache_associativity="8" replacement_protocol="LRU"/>
             <Cache depth="1" cache_size="32768" cache_linesize="64" cache_associativity="8" replacement_protocol="LRU"/>
           </Cache>
         </Cache>
       </Architecture>
AUTHORS
       Written by the CASSIS team at ENSEIRB-Matmeca, FRANCE. The team was composed of Nicolas Dubois, Pierre Goudet, Nicolas Heng, Alexandre Horonat, Gilles Marait, Gregoire Pichon.

SEE ALSO
       cassis(1), lstopo(1)

CASSIS 1.0.0                                                                                                  12/03/2014                                                                                                     CASSIS(7)
\end{lstlisting}


\chapter{Dessins des automates}
\section{Légende et normes communes}

\section{MSI}

\section{MESI}

\section{MOSI}

\section{MESIF}

\section{MOESI}

\chapter{Tutoriel d'ajout d'une politique de cohérence}
\section{Ajout d'une politique de remplacement}
Les politiques de remplacement mise en {\oe}uvre dans \textsf{Cassis} sont classiques. Il est envisageable de rajouter des politiques de remplacement, qui par exemple utiliserait plus de statistiques en interne pour mieux choisir les données à évincer.

Grâce à l'utilisation de pointeurs de fonctions, la gestion de la politique de remplacement au sein d'un cache se fait de façons automatique dans la partie du code qui effectue les actions à réaliser en cas de \emph{load} ou \emph{store}. Il suffit d'ajouter dans la structure cache les bons pointeurs de fonctions pour deux fonctions: \\
\begin{itemize}
\item update\char`_lines\char`_POLICY(block, nb\char`_ways, entry)
\item id\char`_line\char`_to\char`_replace\char`_POLICY(block, priorité, not\char`_rm) \\
\end{itemize}
La première sert à mettre à jour les flags de remplacement relatifs à la nouvelle politique, alors que la deuxième sert à identifier la ligne à supprimer lorsqu'un cache est plein. A noter que les paramètres priorité et not\char`_rm servent à améliorer le fonctionnement global des politiques de cohérence. La priorité est utilisée dans le cas de l'usage de \emph{directory manager} pour identifier les données qui sont présentes dans les caches en dessous alors que le paramètre not\char`_rm sert à identifier une donnée qui va très probablement être ajoutée dans un futur proche. Ces deux paramètres sont toutefois facultatifs, ils n'empêchent pas de fournir des résultats cohérents, et leur utilisation peut être inutile pour une nouvelle politique. \\

Il faut donc créer ces duex fonctions dans le fichier block.c, puis ajouter une function de prototype void replacement\char`_POLICY(cache) dans cache.c. Il faut pour finir permettre d'ajouter la politique au fichier de configuration en XML. Pour cela, il suffir de rajouter, dans la fonction get\char`_replacement\char`_function(cache), présente dans architecture.c, une entrée vers la nouvelle politique. \\

Pour ajouter une politique de remplacement plus globale permettant par exemple d'obtenir des informations à partir d'autres caches, il faut modifier le prototype de fonction pour prendre un n{\oe}ud à la place d'un block. Il faudra cependant s'assurer de modifier toutes les fonctions existantes pour les $3$ politiques déjà implémentées.

\section{Ajout d'une politique de cohérence}
\label{tuto_aut}

\subsection{Sources à modifier}

Les politiques de cohérence sont toutes décrites dans un seul fichier, écrit grâce à \textsf{SMC} : \texttt{coherence.sm}. Toutefois d'autres sources sont à modifier afin que ces politiques puissent être choisies depuis un fichier d'architecture.

\subsubsection{\texttt{common\_types.h}}

\subsubsection{\texttt{architecture.c}}

\subsubsection{\texttt{coherence.c}}

\subsubsection{\texttt{coherence.sm}}

\subsection{Actions possibles}

\subsubsection{Usage de \textsf{SMC}}

\subsubsection{Uage des fonctions du simulateur}

\nocite{*}
\bibliographystyle{plain}
\bibliography{rapport}

\end{document}
