Un simulateur présente l'intérêt de se dégager de certaines contraintes imposées aux expériences sur des systèmes réels. La simulation ne serait donc que peu utile s'il existait trop de contraintes. La forte paramétrisation du simulateur \textsf{Cassis} était donc un objectif impératif, introduit dès la phase de spécification des besoins. Il convient d'avoir la possibilité de simuler des programmes sur des architectures et des politiques de cohérence et d'entrelacement descriptibles facilement. Des langages spécifiques ont été utilisés pour chacun des aspects, afin soit de permettre une modification rapide des paramètres de simulation sans recompilation, soit de permettre une modification dans un langage adapté au problème.

\section{Paramétrisation de l'architecture}

\label{param_xml}
L'architecture à simuler peut être générée à partir de l'architecture réelle de l'utilisateur au moyen d'un fichier XML créé par le logiciel \textsf{HWLOC}. Cependant l'utilisateur peut utiliser un fichier de paramétrisation spécifique à notre simulateur qui lui permet d'accéder à l'intégralité des paramètres d'architecture pris en compte.

\subsection{Entrée XML HWLOC}

\textsf{HWLOC} est un logiciel libre sous licence BSD-2. Il permet de générer un fichier XML qui décrit l'architecture de la machine utilisée (commande \verb?lstopo --of xml?). Il décrit notamment la structure arborescente des caches, et donne des informations essentielles pour chaque cache, comme sa taille, la taille de ses lignes et son associativité. 

\paragraph{}
Si l'utilisateur choisit un tel fichier en entrée comme décrivant son architecture, ce dernier sera parsé par une feuille \textit{xslt} en un fichier de configuration de l'architecture personnalisé, comme décrit dans la section \ref{config}. Les paramètres non fournis par le fichier généré par \textsf{HWLOC} prendront des valeurs par défaut, proches de celles des architectures \textsf{Intel} moderne (voir l'exemple en annexe \ref{manarchi}). Cette organisation est décrite par un schéma figure \ref{img:archi}. Notons que notre simulateur ne prend pas en compte les caches de niveau 1 dédiés aux instructions (L1i), qui sont décrits par \textsf{HWLOC} mais ne seront pas présent dans le fichier personnalisé.

\begin{figure}[!h]
\begin{center}
   \includegraphics[width=0.7\textwidth]{images/schema_archi.png}
   \caption{\label{img:archi} Fichiers de paramétrisation de l'architecture}
\end{center}
\end{figure}

\subsection{Fichier de configuration personnalisé}
\label{config}
Le fichier de configuration de l'architecture dédié à notre simulateur comprend tous les paramètres d'achitecture utilisables. Une fois généré à partir d'un fichier \textsf{HWLOC}, il est possible de l'utiliser directement en entrée du simulateur, après avoir été modifié à la convenance de l'utilisateur. La documentation précise ainsi qu'un exemple de fichier \textit{.cassis.xml} est disponible dans une page de manuel, donnée en annexe \ref{manarchi} page \pageref{manarchi}.

\paragraph{}
Il s'agit d'un fichier XML qui contient 3 balises : \textbf{Architecture}, \textbf{Level} et \textbf{Cache}.

\paragraph{Architecture} donne les infomations générales sur l'architecture, telles que le son nom, le modèle du microprocesseur et le nombre de niveaux de cache.

\paragraph{Level} décrit les informations relatives à un niveau de cache en particulier. Il possède notamment les attributs suivants :
\begin{itemize}
  \item \textbf{coherence\_protocol} : MESI, MSI, MOSI, MESIF, ou MOESI (cf. \ref{coherence} page \pageref{coherence}), qui sont les protocoles implémentés pour le simulateur. Le dernier niveau ne possède pas de protocole de cohérence, car il est le seul cache dans son niveau.
  \item \textbf{type} : l'inclusivité des caches du niveau (cf. \ref{inclusivite} page \pageref{inclusivite}). Ce paramètre n'a pas de sens pour les L1.
  \item \textbf{snooping} : y a-t-il du \emph{snooping} à ce niveau ? (cf. \ref{snooping} page \pageref{snooping}). Le cache du dernier niveau ne peut pas faire de \emph{snooping}.
  \item \textbf{directory\_manager} : les caches du niveau possèdent-il un \emph{directory manager} ? Ce paramètre n'est pas pris en compte pour les L1.
\end{itemize}

\paragraph{Cache} concerne les informations spécifiques à un cache en particulier. Il donne notamment la taille en octets du caches, la taille d'une ligne de cache, l'associativité du cache, et enfin le protocole de cohérence du cache(FIFO, LRU ou LFU, cf. \ref{remplacement} page \pageref{remplacement}).

\paragraph{}
Il est ainsi possible de simuler un bon nombre d'architectures, même si certaines n'ont pas de sens. Le simulateur effectue une vérification sur l'achitecture avant la simulation, afin d'écarter certaines architectures qui ont de grandes chances de faire planter le programme. Nous avons néanmoins laissé à l'utilisateur la possibilité de passer outre ces vérifications.



\section{Entrelacement des threads}

Les traces simulées sont générées par \textsf{MAQAO}, elles représentent une suite linéaire d'instructions. Elles ne reflètent donc pas le comportement réel de l'exécution parallèle du programme où les instructions sont réparties entre les différents c\oe ur du processeur. 

\subsection{Nécessité de l'entrelacement}

Sur une architecture multi-coeur plusieurs threads sont exécutés à la fois. Les threads exécutent donc des instructions en parallèle permettant ainsi d'augmenter la vitesse de calcul. Le problème est qu'il est possible de savoir sur quel c\oe ur une instruction a été exécutée, mais pas sa position dans l'exécution global du programme. En effet un c\oe ur a pu recevoir un appel système et temporairement exécuter d'autres opérations que celles du programme simulé, l'ordre d'exécution des instructions sur tous les c\oe ur ne peut donc pas a priori être connu. 

\paragraph{}
Un modèle de base a donc été implémenté afin de configurer l'ordre d'exécution des traces, appelé entrelacement des threads. Celui choisi pour le simulateur fonctionne de la manière suivante : un certain nombre d'instructions sont exécutées sur un thread, une fois terminées ce sont autant d'instructions qui sont exécutées sur le c\oe ur suivant (modulo le nombre de c\oe urs, car c'est un cycle). Cela revient à séquencialiser le programme en entrelaçant les instructions sur les différents threads.

\subsection{Entrelacement bloc par bloc}

Le choix d'entrelacement qui a été fait permet donc de changer de coeur sur lequel il faut exécuter les instructions de manière cyclique. La technologie utilisée est un script LUA qui retourne le prochain coeur sur lequel le simulateur doit exécuter les instructions. Ce script permet de fixer un nombre d'instructions à exécuter avant de changer de coeur, pour changer de coeur à chaque instructions il faut donc réduire la taille du bloc d'instructions à exécuter à une seule instruction.\\

===============> Joli schéma de l'exécution bloc par bloc


\paragraph{}
Le modèle d'entrelacement des threads étant placé dans un fichier LUA, celui-ci est modifiable sans nouvelle compilation. Dans le cas où les traces n'ont pas toutes le même nombre d'instructions, il n'est toutefois pas possible depuis le script de savoir si une trace est terminée alors que les autres non. Si le c\oe ur renvoyé par le script n'a plus d'instruction à lire sur la trace associée, rien ne se produira, et le script renverra le c\oe ur suivant. 


\begin{frame}{Nécessité de la cohérence}
    \includegraphics[scale=.3]{images/learn_mesi_2.png}  
\end{frame}

\begin{frame}{Solution MESI}
  \begin{block}{Cohérence = automate}
    Les lignes de caches respectent certaines caractéristiques d'un automate : ont un nombre fini d'états, et passent de l'un à l'autre suivant des actions extérieurs (i.e. étiquettes/transitions).\\
    Pour \textsf{Cassis}, usage de \textsf{SMC} afin de représenter les automates de cohérences.
  \end{block}
  
  \begin{block}{Les états :}
    \begin{itemize}
    \item{I (donnée invalide),}
    \item{M (donnée modifiée),}
    \item{S (donnée partagée),}
    \item{E (donnée unique).}
    \end{itemize}
  \end{block}
\end{frame}

\begin{frame}{Automate MESI : les transitions}
    \includegraphics[scale=.3]{images/MESI_simple.png}
\end{frame}

\begin{frame}{Autres protocoles}
  \begin{block}{Protocoles implémentés}
    Les autres protocoles courrants sont :
    \begin{itemize}
    \item{MSI (le plus basique),}
    \item{MOSI/MESIF (alternative presqu'équivalente à MESI),}
    \item{MOESI (le plus complexe).}
    \end{itemize}
  \end{block}
  
  \begin{block}{Avantages/Inconvénients}
    Deux critères importants pour le choix d'un protocole :
    \begin{itemize}
    \item{le nombre de \emph{write-back},}
    \item{l'utilisation de la bande-passante.}
    \end{itemize}
  \end{block}
\end{frame}


