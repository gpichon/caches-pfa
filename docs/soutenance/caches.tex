\begin{frame}{Hiérarchie mémoire}
  \begin{columns}[c]
    \begin{column}{.5\textwidth}
      \begin{figure}[h!]
	\includegraphics[scale=.35]{images/hierarchy.png}
      \end{figure}
    \end{column}
    \begin{column}{.5\textwidth}
      \begin{block}{Objectif}
	\begin{itemize}
	\item{Trouver un bon compromis entre vitesse et coût}
	\end{itemize}
      \end{block}
    \end{column}
  \end{columns}
  \begin{columns}[c]
    \begin{column}{.6\textwidth}
      \begin{block}{Hiérarchie de caches}
	\begin{itemize}
	\item{Plusieurs niveaux de cache}
	\item{Caches inclusifs/exclusifs}
	\end{itemize}
      \end{block}
    \end{column}
    \begin{column}{.4\textwidth}
      \begin{figure}[h!]
	\includegraphics[scale=.3]{images/lstopo.png}
      \end{figure}
    \end{column}
  \end{columns}
\end{frame}

\begin{frame}{Ajout d'une donnée dans un cache (1)}  
  \begin{block}{Organisation d'un cache}
    Cache contenant plusieurs lignes, référencées par des étiquettes.
  \end{block}
  \begin{center}
  \includegraphics[scale=.35]{images/etiquette.jpeg}
  \end{center}

  \begin{block}{Recherche de la donnée}
    \begin{itemize}
    \item{Si le c\oe ur trouve la donnée recherchée dans le cache : il fait un \emph{hit}.}
    \item{Sinon, la donnée est rapatriée à partir de la mémoire de niveau supérieur : c'est un \emph{miss}.}
    \end{itemize}
  \end{block}
\end{frame}	

\begin{frame}{Associativité}
  \begin{block}{Fonction de correspondance adresse mémoire / cache}
    \begin{itemize}
    \item{Direct associative}
    \item{Fully associative}
    \item{k-ways associative}
    \end{itemize}
  \end{block}
  \begin{figure}[h!]
    \includegraphics[scale=.33]{images/associative.png}
  \end{figure}
\end{frame}

\begin{frame}{Ajout d'une donnée dans un cache (2)}
  \begin{block}{Rapatriement de la donnée : cas du \emph{set} plein}
    Il existe différentes politiques de remplacement :
    \begin{itemize}
    \item{FIFO (Supprimer la plus ancienne)}
    \item{LFU (Supprimer la moins utilisée)}
    \item{LRU (Supprimer la plus anciennement utilisée)}
    \end{itemize}
    En général, les données évincées sont déplacées vers la mémoire de niveau supérieur.
  \end{block}
  \begin{block}{Réécriture des données en cas de modification}
    \begin{itemize}
    \item{\emph{Write-through} : Une donnée modifiée est tout de suite reportée aux niveaux supérieurs.}
    \item{\emph{Write-back} : Le report est fait au plus tard.}
    \end{itemize}
  \end{block}

\end{frame}
