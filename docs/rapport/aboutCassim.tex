\chapter{Simulateur de caches : \textsc{Cassim}}

En vrac, non trié.

===> Rajouter une intro de partie rappelant la nécéssité d'un simulateur ? (par manque de connaissance/documentation dans le domaine, etc...)
===> Enchaîner ensuite sur le fait qu'on doit pouvoir choisir les paramètres de simulation (pour pouvoir simuler des architectures variés, et même expérimentaux ou pour obtenir le suivi de certaines données particulières)
===> Comment on a implémenté cette entre-guillemets "modularité" -- comment l'utilisateur peut choisir ses paramètres (script Lua pour l'entrelacement, XML c'est déjà fait, etc...)
===> D'autres trucs sur l'implém, par exemple préciser ce que ne fait pas le simulateur (calcul de bande passante, prise en compte du prefetching) ???
===> Exemple d'entrée/sortie, comparaison de la sortie avec ceux obtenus avec d'autres logiciels (compteur hard) + graphes en fonction de la taille de l'entrée (pour voir s'il n'y a pas de déviation lorsqu'on a des programmes plus complexes)

\section{Algorithmes de simulation}
Cette section entend présenter différents algorithmes utilisés par notre simulateur. Le but du simulateur est de produire des statistiques à partir d'une trace des blocs mémoires utilisés (\textit{load}/\textit{store}). Les données utilisées ne sont pas simulées. Le comportement du simulateur peut donc varier de la réalité, pourvu que le résultat final soit identique.

\subsection{Gestion de la cohérence}
La cohérence étant propre à un niveau de cache, elle peut être mise en place directement en considérant l'ensemble des caches de ce niveau. En réalité, les L1 ou les L2 n'intéreagissent pas toujours directement: s'ils ne sont pas reliés par un bus (cas du snooping), il faut passer par les niveaux haut dessus pour envoyer les messages. Cependant, la quantité de messages envoyés par cache pouvant être retrouvée à partir du nombre de \textit{misses} et de \textit{hits}, il n'est pas nécessaire de la calculer. \\

L'algorithme de cohérence (MSI, MESI, MOSI ou MOESI) est donc relativement facile à implémenter. Dans le cas du protocole MESI, il y a deux cas majeurs à gérer: \\
\begin{itemize}
\item Lorsqu'un cache fait un \textit{miss}, il parcours l'ensemble des autres caches pour savoir s'il doit mettre la donnée dans l'état E ou S. Dans le cas S, il modifie les données des autres caches pour quelles soient dans l'état S.
\item Quand un cache modifie une donnée qu'il avait dans l'état S, il invalide la donnée dans les autres caches.
\end{itemize}

\subsection{Interaction entre les différents niveaux de caches}

\subsection{Gestion des différentes modularités}

\input{entreeXML.tex}
