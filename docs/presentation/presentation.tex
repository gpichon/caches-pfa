\documentclass{beamer}
\usepackage[utf8]{inputenc}
\usepackage[T1]{fontenc}
\usepackage[french]{babel}

\usepackage{amsmath}
\usepackage{amssymb}
\usepackage{color}
\usepackage{graphicx}

\usetheme{CambridgeUS}
\title[Projet de programmation orientée objet]{Collecte de données}
\author[Les Manchots]{Nicolas Belin, Loïc Dauphin, Yvon Garbage, \\ Pierre Lefebvre, Grégoire Pichon}
\institute[ENSEIRB-MATMECA]{ENSEIRB-MATMECA}
\date{\today}
%\date{}

\setbeamercolor{title}{bg=red!65!black,fg=white}

\begin{document}

\setlength{\unitlength}{1cm}

\begin{frame}{Présentation}

\titlepage

\end{frame}

%\AtBeginSection[]
%{
%\begin{frame}<beamer>
%  \frametitle{Plan}
%  \tableofcontents[currentsection]
%\end{frame}
%}

\section*{Introduction}
\begin{frame}
\begin{block}{Sujet}
~~Service de collecte de l'entrée et de la sortie des passagers\\
\end{block}

\begin{block}{Objectifs}
\begin{center}
\begin{itemize}
  \item{indépendance du service de collecte,}
  \item{utilisation de la collecte facultative pour le client,}
  \item{indépendance des réalisations du service de collecte,}
  \item{limitation de la duplication de code.}
\end{itemize}
\end{center}
\end{block}
\end{frame}

\section{Implémentation}
\subsection{Cadre général}
\begin{frame}
\begin{block}{Code}
\begin{center}
\begin{itemize}
\item utilisation d'une LinkedList<int$\left[~\right]$>,
\item à chaque nouvel arrêt, on enregistre les entrées et les sorties dans un nouvel élément de la liste,
\item Collecte instanciée avec une chaîne de caractères,
\item un fichier de log par collecte instanciée.
\end{itemize}
\end{center}
\end{block}
\end{frame}

\subsection{Greffon est-un}
\begin{frame}
\begin{center}
\begin{figure}
\includegraphics[scale=0.35]{diagramme-classe-est-un.png}
\end{figure}
Diagramme est-un
\end{center}
\end{frame}

\begin{frame}
\begin{center}
\begin{block}{Code est-un}
\begin{itemize}
\item \texttt{AutobusCollecte extends Autobus}
\item instanciation de AutobusCollecte : \texttt{(int assis, int debout, Collecte)},
\item code factorisé grâce à l'héritage des méthodes d'Autobus.
\end{itemize}
\end{block}
\begin{figure}
\includegraphics[scale=0.32]{diagramme-sequence-est-un.png}
\end{figure}
\end{center}
\end{frame}

\subsection{Greffon a-un}
\begin{frame}
\begin{center}
\begin{figure}
\includegraphics[scale=0.35]{diagramme-classe-a-un.png}
\end{figure}
Diagramme a-un
\end{center}
\end{frame}

\begin{frame}
\begin{center}
\begin{block}{Code a-un}
\begin{itemize}
\item \texttt{AutobusCollecte extends Bus implements Transport}
\item lien a-un : AutobusCollecte possède un attribut de type Transport,
\item instanciation de AutobusCollecte: \texttt{(Transport, Collecte)},
\item réutilisation possible avec une autre classe descendant de Transport et Bus, par exemple Tramway,
\item définition des méthodes déclarées dans l'interface Transport et la classe abstraite Bus.
\end{itemize}
\end{block}
\begin{figure}
\includegraphics[scale=0.3]{diagramme-sequence-a-un.png}
\end{figure}
\end{center}
\end{frame}

\section*{Conclusion}
\begin{frame}
\begin{block}{Conclusion}
\begin{center}
\begin{itemize}
\item améliorations possibles de la collecte : interface Collecte,
\item fermeture des fichiers réalisée avec la méthode finalize().
\end{itemize}
\end{center}
\end{block}
\end{frame}

\end{document}
